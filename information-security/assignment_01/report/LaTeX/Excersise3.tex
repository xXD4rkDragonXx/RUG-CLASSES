\documentclass{report}
\usepackage[utf8]{inputenc}
\usepackage{hyperref}

\title{Information Security Assignment 3}
\author{
    Stijn Kammer (S4986296)
    \and Ramon Kits (S5440769)
}
\date{September 2022}

\begin{document}

\section*{Exercise 3}

\begin{enumerate} 
    \item \textbf{How many possible alphabets could be used in a substitution cipher that uses shifting? What about the number of possible alphabets in a substitution cipher that uses a mixed alphabet?}
        For a substitution cipher that uses shifting, there are 26 possible alphabets. all possible combinations of the alphabet with a modulus 26. 
        For a substitution cipher that uses a mixed alphabet, there are 26! possible alphabets. All possible permutations of the alphabet.
    
    \item \textbf{Does applying a significant number of consecutive simple substitution cipher encryptions/decryptions with a mixed or shifted alphabet make it harder to break the original plaintext? Justify your answer.}
        No it does the opposite. The more times you apply a simple substitution cipher, the more times you apply the same substitution. This means that the ciphertext will be more and more similar to the plaintext. This makes it easier to break the original plaintext.

    \item \textbf{Can the encryption function of the substitution cipher also be used for decryption? If so, how?}
        Yes, the encryption function of the substitution cipher can also be used for decryption. By running the encryption function with the same key multiple times, the plaintext will eventualy be returned. 
        This means that the encryption function can be used for decryption.