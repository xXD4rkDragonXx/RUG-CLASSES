\documentclass[twoside, a4paper, fleqn, reqno]{article}
\usepackage[
	assignmentNumber=4, 
	authorZero={Stijn Kammer},
	studentNumberZero={4986296},	
	authorOne={Ramon Kits},
	studentNumberOne={5440769},
	groupNumber={31}
]{reportStyle}


\begin{document}

\maketitle

\section*{Exercise 1}
\begin{enumerate}
	\item\textbf{Consider a CRC that uses the divisor 10011. Find two collisions with
	10101011, that is, find two other data values that produce the same
	CRC checksum as 10101011.}
	\\First we have to compute the CRC checksum of \texttt{10101011} using the divisor \texttt{10011}.
	\\\texttt{\phantom{}10101011\color{red}0000}
	\\\texttt{\phantom{}\underline{10011~~}~~~~~}
	\\\texttt{\phantom{}~~11001~~~~~}
	\\\texttt{\phantom{}~~\underline{10011~}~~~~}
	\\\texttt{\phantom{}~~~10101~~~~}
	\\\texttt{\phantom{}~~~\underline{10011~~}~~}
	\\\texttt{\phantom{}~~~~~110\color{red}00~~}
	\\\texttt{\phantom{}~~~~~\underline{100\color{red}11~}~}
	\\\texttt{\phantom{}~~~~~~10\color{red}110~}
	\\\texttt{\phantom{}~~~~~~\underline{10\color{red}011~}}
	\\\texttt{\phantom{}~~~~~~~~\color{red}1010}
	\\The CRC checksum is therefore 1010.
	Now we have to find two other data values that produce the same CRC checksum.
	The first place where zeros have been added has to be the same in order to
	find a collision. In this case, we have to find a value that produces a 
	step where \texttt{110\color{red}00} is used. That is, the first step where
	CRC bits are being added. So to make it easy to find a collision,
	we can just use \texttt{110} as value because this generates the step we need instantly.
	Also, we can use one of the other intermediate steps to find a collision, e.g. \texttt{10101}.
	This way of finding collisions is very limited, only steps of the original value can be used.
	To find a collision with a value of the same length as the original, we can also
	use the divisor as the first part of the value so when a XOR is performed on that part,
	the first 5 bits will get discarded in a sense.
	At this point, we are left with 3 bits that have to be \texttt{110} to get the desired step we want.
	So the value we end up with is \texttt{10011110}.\\
	We now have found three collisions using two different methods.
	Collisions can also be found in more sophisticated ways, but with the simplicity of the question
	in mind, only two ways are discussed here. Our found values are
	\textbf{\underline{\texttt{110}}}, \textbf{\underline{\texttt{10101}}} and \textbf{\underline{\texttt{10011110}}}.
	Which are more than two, but it makes it possible to demonstrate multiple ways of finding collisions.
	

	\item\textbf{Consider a CRC that uses the divisor 10011. Suppose the data value
	is 11010110. Trudy wants to change the data to 111*****, where "*"
	indicates that she doesn't care about the bit in that position, and she
	wants the resulting checksum to be the same as for the original data.
	Determine all data values Trudy could choose.}
	For this question, we have to find all data values that Trudy could choose.
	Now we cannot use the same method as in the previous question because we want 
	all possible data values that start with \texttt{111} and have a length of 8 bits.
	Trudy has the opportunity to generate any of $2^5=32$ possible values for the remaining 5 bits,
	of which only a select amount create the same CRC checksum as the original data value.
	Brute forcing all options is possible but not scalable. So we have to find a way to
	compute the possible values without having to check all of them.	
	\\We have to solve for the unknown bits in the following checksum calculation steps.
	\\\texttt{\phantom{}111*****\color{red}0000}
	\\\texttt{\phantom{}\underline{10011~~}~~~~~}
	\\\texttt{\phantom{}~11...~~~~~~~~}
	\\\texttt{\phantom{}~\underline{10011}~~~~~~~~}
	\\\texttt{\phantom{}~~1...~~~~~~~}
	\\\texttt{\phantom{}~~\underline{10011}~~~~~~}
	\\\texttt{\phantom{}~~~...~~~~~~}
	\\\texttt{\phantom{}~~~\underline{...~~}~~~~}
	\\\texttt{\phantom{}~~~~~110\color{red}00~~}
	\\\texttt{\phantom{}~~~~~\underline{100\color{red}11~}~}
	\\\texttt{\phantom{}~~~~~~10\color{red}110~}
	\\\texttt{\phantom{}~~~~~~\underline{10\color{red}011~}}
	\\\texttt{\phantom{}~~~~~~~~\color{red}1010}
	\\For the one step before we reach \texttt{110\color{red}00}, we can say that there are a small
	group of possibilities for the unknown bits.
	\\We can fill in the empty step in the only possible way, and we can remove that step all together.
	That gives us the two only solutions.
	\\Filling these values in the checksum calculation steps, we get the following results.
	\\\texttt{\phantom{}11100111\color{red}0000~~~}					\texttt{\phantom{}~~~~~11110100\color{red}0000}
	\\\texttt{\phantom{}\underline{10011~~}~~~~~~~~}				\texttt{\phantom{}~~~~~\underline{10011~~}~~~~~}
	\\\texttt{\phantom{}~11111~~~~~~~~~}							\texttt{\phantom{}~~~~~~11011~~~~~~~~}
	\\\texttt{\phantom{}~\underline{10011}~~~~~~~~~}				\texttt{\phantom{}~~~~~~\underline{10011}~~~~~~~~}
	\\\texttt{\phantom{}~~11001~~~~~~~~}							\texttt{\phantom{}~~~~~~~10000~~~~~~}
	\\\texttt{\phantom{}~~\underline{10011}~~~~~~~~}				\texttt{\phantom{}and~~~~\underline{10011}~~~~~~}
	\\\texttt{\phantom{}~~~10101~~~~~~~}							\texttt{\phantom{}~~~~~~~~~~110\color{red}00~~}
	\\\texttt{\phantom{}~~~\underline{10011}~~~~~~~}				\texttt{\phantom{}~~~~~~~~~~\underline{100\color{red}11~}~}
	\\\texttt{\phantom{}~~~~~110\color{red}00~~~~~}					\texttt{\phantom{}~~~~~~~~~~~10\color{red}110~}
	\\\texttt{\phantom{}~~~~~\underline{100\color{red}11~}~~~~}		\texttt{\phantom{}~~~~~~~~~~~\underline{10\color{red}011~}}
	\\\texttt{\phantom{}~~~~~~10\color{red}110~~~~}					\texttt{\phantom{}~~~~~~~~~~~~~\color{red}1010}
	\\\texttt{\phantom{}~~~~~~\underline{10\color{red}011~}~~~}		
	\\\texttt{\phantom{}~~~~~~~~\color{red}1010~~~}					
	\\The only two answers are \textbf{\underline{\texttt{11100111}}} and \textbf{\underline{\texttt{11110100}}}. 

\end{enumerate}

\section*{Exercise 2}



\section*{Exercise 3}

\section*{Exercise 4}

\section*{Exercise 5}

\section*{Exercise 6}

\section*{Exercise 7}

\section*{Exercise 8}

\section*{Exercise 9}

\section*{Exercise 10}

\end{document}