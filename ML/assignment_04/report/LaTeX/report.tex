\documentclass[twoside, a4paper, fleqn, reqno]{article}
\usepackage[
	assignmentNumber=4, 
	authorZero={Stijn Kammer},
	studentNumberZero={4986296},	
	authorOne={Ramon Kits},
	studentNumberOne={5440769},
	groupNumber={31}
]{reportStyle}

\begin{document}

\maketitle

\section{introduction}

\section{Methods}

Vector Quantization makes use of prototype vectors to represent the data and is often used for identification and grouping of clusters in similar data.
To use Vector Quantization, we first need to define a set of prototype vectors $\mathbf{v}_1, \mathbf{v}_2, \ldots, \mathbf{v}_k$.
After which present a single example $\mathbf{x}$, we can find the closest prototype vector $\mathbf{v}_i$ by minimizing the distance between $\mathbf{x}$ and $\mathbf{v}_i$.
The distance between $\mathbf{x}$ and $\mathbf{v}_i$ is defined as the Euclidean distance between the two vectors.
When the closest prototype vector $\mathbf{v}_i$ is found, we move the prototype vector $\mathbf{v}_i$ towards $\mathbf{x}$ by a small amount $\eta$.
Repeat this process for all the examples in the training set and the prototype vectors will converge to a set of vectors that represent the data well.
The prototype vectors are then used to classify new examples.

\section{Learning curves}

\section{Trajectories of prototypes}

\section{Discussion}

\section{Bonus}

\end{document}