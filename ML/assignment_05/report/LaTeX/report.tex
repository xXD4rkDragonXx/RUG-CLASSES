\documentclass[twoside, a4paper, fleqn, reqno]{article}
\usepackage[
	assignmentNumber=5, 
	authorZero={Stijn Kammer},
	studentNumberZero={4986296},	
	authorOne={Ramon Kits},
	studentNumberOne={5440769},
	groupNumber={31}
]{reportStyle}
\date{September 19, 2022}

\begin{document}

\maketitle

\section*{Introduction}
Learning Vector Quantization (LVQ) is a prototype-based supervised classification algorithm.
LVQ can be trained on a dataset with labels and then used to classify new data without labels.
It can be compared to Vector Quantization (VQ), which is an unsupervised algorithm
that is used for data compression and finding clusters in data. LVQ works by
defining a set of prototypes and then moving them to better represent the data.
The prototypes are moved in the direction of the input vector if the input vector
belongs to the class of the prototype, and in the opposite direction if it does not.
After several iterations, the prototypes should have moved to a position where
they represent the data well. The prototypes are then used to classify new data.
There are many different variations of LVQ, but the most common and easy to understand is LVQ1.

\section*{Methods}

\subsection*{LVQ1}
	For this assignment, the LVQ algorithm has been implemented in Python.
	For the majority of the assignment, the standard 2-dimensional LVQ dataset has been
	used which was delivered by the assignment.\\
	The LVQ1 algorithm has been implemented in the following steps:
	\begin{enumerate}
		\item The dataset is loaded and split into 2 classes. The first 50 samples are
			class 0 and the last 50 samples are class 1.
			\begin{tabbing}
				\lstinputlisting[
					language=python,
					firstline=13,
					lastline=18,
					caption={dividing classes},
					label={lst:1:lvq}
					]{../../code/lvq.py}			
			\end{tabbing}
		\item The prototypes are initialized. The prototypes are initialized randomly
			within the range of the dataset.
			\begin{tabbing}
				\lstinputlisting[
					language=python,
					firstline=20,
					lastline=39,
					caption={initializing prototypes},
					label={lst:2:lvq}
					]{../../code/lvq.py}			
			\end{tabbing}
			First the data is sorted by class if it not already is. Then the prototypes are
			initialized by randomly selecting a sample from a subset of a class.
			Subsets are equally divided over the prototypes of that class. This makes it less
			likely that the prototypes are initialized in the same area. This code also supports
			prototypes to be assigned to the mean vector of a class.
		\item Start the first epoch, at the start of each epoch the prototypes are shuffled.
		\item Loop through all samples in the dataset.
		\item Get the closest prototype to the sample. In this case, the list of prototypes is
			sorted by distance to the sample.
			\begin{tabbing}
				\lstinputlisting[
					language=python,
					firstline=86,
					lastline=86,
					caption={getting the closest prototype},
					label={lst:3:lvq}
					]{../../code/lvq.py}
			\end{tabbing}
			The first prototype in the list is the closest so it is returned.
		\item If the sample belongs to the class of the prototype, move the prototype in the
			direction of the sample. If the sample does not belong to the class of the prototype,
			move the prototype in the opposite direction of the sample.
			\begin{tabbing}
				\lstinputlisting[
					language=python,
					firstline=148,
					lastline=153,
					caption={moving the prototype},
					label={lst:4:lvq}
					]{../../code/lvq.py}
			\end{tabbing}
		\item If the sample is the last in the dataset, start the next epoch.
		\item When the maximum number of epochs is reached or the Quantization Error is roughly
			stable, stop the algorithm.
			\begin{tabbing}
				\lstinputlisting[
					language=python,
					firstline=167,
					lastline=177,
					caption={stopping the algorithm},
					label={lst:5:lvq}
					]{../../code/lvq.py}
			\end{tabbing}
			The above code shows the stopping criteria. The Quantization Error is calculated by
			summing the wrong classifications and dividing that by the total number of samples.
			The Quantization Error is calculated at the end of each epoch. From errors,
			a moving average is calculated. The moving average is used to determine if the
			Quantization Error is roughly stable. When the moving average has not changed
			more than $1\%$ in the last $x$ epochs, the algorithm is stopped.
	\end{enumerate}
\subsection*{GLVQ}
	For the GLVQ algorithm, the same steps as for the LVQ1 algorithm have been taken.
	The only difference is that when moving the prototypes, not only the closest prototype
	is used, but also the second closest prototype from a different class.
	The prototypes are moved in the direction of the sample if the sample belongs to the
	class of the closest prototype, and in the opposite direction if it does not.
	\begin{tabbing}
		\lstinputlisting[
			language=python,
			firstline=135,
			lastline=141,
			caption={moving the prototype},
			label={lst:6:lvq}
			]{../../code/lvq.py}
	\end{tabbing}
	Above code shows this behavior. The amount the prototypes are moved is determined by
	the learning rate and the distance between the sample and the prototype. When the
	GLVQ algorithm is used, the learinging rate decreases over time. Every epoch, the
	learning rate is lowered by $10\%$ as shown in the code below.
	\begin{tabbing}
		\lstinputlisting[
			language=python,
			firstline=178,
			lastline=180,
			caption={lowering the learning rate},
			label={lst:7:lvq}
			]{../../code/lvq.py}
	\end{tabbing}
\subsection*{Iris dataset}
	The code for LVQ1 has been written with multidimensional datasets in mind. The Iris
	dataset has 4-dimensions, so the code can be used for this dataset as well.
	Only the dataset had to be changed. The dataset had labeled samples, the labels
	which were strings had to be converted to integers.

\section*{Results}

\begin{figure}[H]
	\centering
	\subfloat[Original classes with prototypes]{
		\includegraphics[width=0.45\textwidth]{../../code/output/lvq1_2_2_0.002.png}
		\label{fig:lvq1_2_2_0.002}
	}
	\hfill
	\subfloat[New classes with prototypes]{
		\includegraphics[width=0.45\textwidth]{../../code/output/lvq1_2_2_0.002_new_labels.png}
		\label{fig:lvq1_2_2_0.002_new_labels}
	}
	\hfill
	\subfloat[Error percentage per epoch]{
		\includegraphics[width=0.45\textwidth]{../../code/output/lvq1_error_2_2_0.002.png}
		\label{fig:lvq1_error_2_2_0.002}
	}
	\caption{
		These two plots represent LVQ1 with 1 prototype per class, 2 in total, with a learning rate of 0.002.
		The prototypes are shown at their location after the last epoch. The prototypes have been
		randomly initialized from one of the points in the corresponding class.
		\textbf{a} shows the original classes and the trajectories of the prototypes.
		\textbf{b} shows the new classes and the same prototypes the new classification is based on.
		\textbf{c} shows the classification error percentage per epoch and a moving average of 20 epochs.
	}
\end{figure}

\begin{figure}[H]
	\centering
	\subfloat[Original classes with prototypes]{
		\includegraphics[width=0.45\textwidth]{../../code/output/lvq1_4_2_0.002.png}
		\label{fig:lvq1_4_2_0.002}
	}
	\hfill
	\subfloat[New classes with prototypes]{
		\includegraphics[width=0.45\textwidth]{../../code/output/lvq1_4_2_0.002_new_labels.png}
		\label{fig:lvq1_4_2_0.002_new_labels}
	}
	\hfill
	\subfloat[Error percentage per epoch]{
		\includegraphics[width=0.45\textwidth]{../../code/output/lvq1_error_4_2_0.002.png}
		\label{fig:lvq1_error_4_2_0.002}
	}
	\caption{
		These two plots represent LVQ1 with 2 prototypes per class, 4 in total, with a learning rate of 0.002.
		The prototypes are shown at their location after the last epoch. The prototypes have been
		randomly initialized from one of the points in the corresponding class.
		\textbf{a} shows the original classes and the trajectories of the prototypes.
		\textbf{b} shows the new classes and the same prototypes the new classification is based on.
		\textbf{c} shows the classification error percentage per epoch and a moving average of 20 epochs.
	}
\end{figure}

\begin{figure}[H]
	\centering
	\subfloat[Original classes with prototypes]{
		\includegraphics[width=0.45\textwidth]{../../code/output/lvq1_6_2_0.002.png}
		\label{fig:lvq1_6_2_0.002}
	}
	\hfill
	\subfloat[New classes with prototypes]{
		\includegraphics[width=0.45\textwidth]{../../code/output/lvq1_6_2_0.002_new_labels.png}
		\label{fig:lvq1_6_2_0.002_new_labels}
	}
	\hfill
	\subfloat[Error percentage per epoch]{
		\includegraphics[width=0.45\textwidth]{../../code/output/lvq1_error_6_2_0.002.png}
		\label{fig:lvq1_error_6_2_0.002}
	}
	\caption{
		These two plots represent LVQ1 with 3 prototypes per class, 6 in total, with a learning rate of 0.002.
		The prototypes are shown at their location after the last epoch. The prototypes have been
		randomly initialized from one of the points in the corresponding class.
		\textbf{a} shows the original classes and the trajectories of the prototypes.
		\textbf{b} shows the new classes and the same prototypes the new classification is based on.
		\textbf{c} shows the classification error percentage per epoch and a moving average of 20 epochs.
	}
\end{figure}

\begin{figure}[H]
	\centering
	\subfloat[Original classes with prototypes]{
		\includegraphics[width=0.45\textwidth]{../../code/output/lvq1_8_2_0.002.png}
		\label{fig:lvq1_8_2_0.002}
	}
	\hfill
	\subfloat[New classes with prototypes]{
		\includegraphics[width=0.45\textwidth]{../../code/output/lvq1_8_2_0.002_new_labels.png}
		\label{fig:lvq1_8_2_0.002_new_labels}
	}
	\hfill
	\subfloat[Error percentage per epoch]{
		\includegraphics[width=0.45\textwidth]{../../code/output/lvq1_error_8_2_0.002.png}
		\label{fig:lvq1_error_8_2_0.002}
	}
	\caption{
		These two plots represent LVQ1 with 4 prototypes per class, 8 in total, with a learning rate of 0.002.
		The prototypes are shown at their location after the last epoch. The prototypes have been
		randomly initialized from one of the points in the corresponding class.
		\textbf{a} shows the original classes and the trajectories of the prototypes.
		\textbf{b} shows the new classes and the same prototypes the new classification is based on.
		\textbf{c} shows the classification error percentage per epoch and a moving average of 20 epochs.
	}
\end{figure}

\begin{figure}[H]
	\centering
	\subfloat[Original classes with prototypes]{
		\includegraphics[width=0.45\textwidth]{../../code/output/lvq1_mean_4_2_0.002.png}
		\label{fig:lvq1_mean_4_2_0.002}
	}
	\hfill
	\subfloat[Error percentage per epoch]{
		\includegraphics[width=0.45\textwidth]{../../code/output/lvq1_mean_error_4_2_0.002.png}
		\label{fig:lvq1_mean_error_4_2_0.002}
	}
	\hfill
	\caption{
		These two plots represent LVQ1 with 2 prototypes per class, 4 in total, with a learning rate of 0.002.
		The prototypes are shown at their location after the last epoch. The prototypes have been
		initialized on the mean of a subset of the corresponding class. The subset is chosen by the amount of prototypes per class.
		\textbf{a} shows the original classes and the trajectories of the prototypes.
		\textbf{b} shows the classification error percentage per epoch and a moving average of 20 epochs.
	}
\end{figure}

\begin{figure}[H]
	\centering
	\includegraphics[width=0.5\textwidth]{../../code/output/lvq1_error_iris_3_0.005.png}
	\label{fig:lvq1_error_iris_3_0.005}
	\caption{
		This chart shows the classification error percentage per epoch and a moving average of 20 epochs for the Iris dataset.
		This is a 4-dimensional dataset with 3 classes. 3 prototypes have been used,
		which have been initialized randomly on one datapoint from a corresponding class.
	}
\end{figure}

\begin{figure}[H]
	\centering
	\subfloat[Original classes with prototypes]{
		\includegraphics[width=0.45\textwidth]{../../code/output/glvq_scatter_4_2_0.002.png}
		\label{fig:glvq_scatter_4_2_0.002}
	}
	\hfill
	\subfloat[New classes with prototypes]{
		\includegraphics[width=0.45\textwidth]{../../code/output/glvq_error_4_2_0.002.png}
		\label{fig:glvq_error_4_2_0.002}
	}
	\hfill
	\caption{
		These two plots represent GLVQ with 2 prototypes per class, 4 in total, with a learning rate of 0.002.
		The prototypes are shown at their location after the last epoch. The prototypes have been
		initialized on the mean of a subset of the corresponding class. The subset is chosen by the amount of prototypes per class.
		\textbf{a} shows the original classes and the trajectories of the prototypes.
		\textbf{b} shows the classification error percentage per epoch and a moving average of 20 epochs.
	}
\end{figure}


\section*{Discussion}

	\subsection*{LVQ1}
	In \autoref{fig:lvq1_2_2_0.002} we see that the prototypes can find their way to their corresponding class.
	\autoref{fig:lvq1_error_2_2_0.002} shows that the error percentage is decreasing rapidly in the first 30 epochs
	from about $38\%$ to $21\%$. From that moment the error percentage is getting stable. The algorithm stopped
	after 175 epochs, when it decided that the error percentage was stable enough.
	This means that around $80\%$ of the data is classified correctly. This is a good result, taking in mind
	only 2 prototypes were used and the data was very noisy.\\

	In \autoref{fig:lvq1_4_2_0.002} we see 2 prototypes per class. In this case, also the datapoint in the bottom right
	corner is classified correctly since it got initialized in that corner.
	\autoref{fig:lvq1_error_4_2_0.002} shows that the error percentage is decreasing from $43\%$ to $23\%$ in the first $60$ epochs.
	The algorithm stopped after about $70$ epochs. It stopped at a higher error percentage than in the previous case with 1 prototype per class.
	This can be explained by the fact that the one purple prototype is initialized in the corner with one purple datapoint.
	Because of that, it causes more quantization errors in that particular corner. On top of that, the other purple prototype
	is the only one that is able to classify the purple datapoints on the left side.\\

	Both scenarios with 3 and 4 prototypes per class do not show better results either,
	see \autoref{fig:lvq1_error_6_2_0.002} and \autoref{fig:lvq1_error_8_2_0.002}.
	This led to believe that the amount of prototypes with this specific dataset is 1 per class.
	Testing very many amounts of prototypes is not needed, this will lead to overfitting the dataset.
	This way it is easy to get an error percentage of $0\%$ but will make classification of new datapoint
	very untrustworthy.\\

	When testing 2 prototypes per class and the prototypes are initialized on the mean of the corresponding class
	as shown in \autoref{fig:lvq1_mean_4_2_0.002}, the results look very promising.
	The chart in \autoref{fig:lvq1_mean_error_4_2_0.002} confirms this. The error percentage is decreasing from $21\%$ to $19\%$ in the first $60$ epochs.
	What also is interesting is that the error percentage starts at an even lower value than when the error percentage becomes stable.
	As you can see, the prototypes all start near the middle of the plot. Apparently, this starting position
	gave a lower error percentage than when the error stabilizes. This is probably because the two classes are
	roughly separated in the middle of the plot. Due to the prototypes being repelled by the points of the other class,
	they are pushed away from the middle of the plot even though for the final classification this is not needed in this case.\\

	\subsection*{Iris dataset}
	\autoref{fig:lvq1_error_iris_3_0.005} shows a starting error percentage of over $11\%$ and a final error percentage of a little below $7\%$.
	Compared to the other dataset, this is a very good result. This is probably because the Iris dataset is very clean.
	It probably has well-defined clusters and the prototypes can find their way to the correct class.
	The prototypes starting at a low error percentage already can be an indication that points in the dataset are most likely already inside a
	well-defined cluster. Having used the Euclidean distance measure has been a reasonable choice for this dataset since all the features
	have about the same scale. When this was not the case, when one feature had a much higher scale than the others, it would have been better
	to use another distance measure.\\

	\subsection*{GLVQ}
	\autoref{fig:glvq_scatter_4_2_0.002} shows that the prototypes can find their way to their corresponding class.
	Because of the rapid decline in learning rate, the prototypes find a stable position after $40$ epochs.
	Since the learning rate at this point has decreased by a factor of $100$, the prototypes are not able to move much anymore.
	so the error percentage will certainly be stable from this point as can be seen in \autoref{fig:glvq_error_4_2_0.002}.
	The stable error percentage is very similar to the one of LVQ1 with 2, 3 and 4 prototypes per class.
	Since the very simple implementation of this algorithm, there is not yet clear evidence that GLVQ is better than LVQ1.

\section*{Work distribution}
	\begin{itemize}
		\item \textbf{Ramon}:
		\begin{itemize}
			\item Mainly worked on the code for the LVQ1 algorithm and the bonus parts.
		\end{itemize}
		\item \textbf{Stijn}:
		\begin{itemize}
			\item Focussed on the analysis of the algorithms and the report.
		\end{itemize}
	\end{itemize}
	When both parts were in the final stages, both team members worked together on the report to make sure everything was correct, clear and understandable.
	This strategy facilitated the sharing of knowledge and the work was divided in a way that both the team members could learn from each other.

\end{document}