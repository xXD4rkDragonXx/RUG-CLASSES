\documentclass[twoside, a4paper, fleqn, reqno]{article}
\usepackage[
	assignmentNumber=3, 
	authorZero={Stijn Kammer},
	studentNumberZero={4986296},	
	authorOne={Ramon Kits},
	studentNumberOne={5440769},
	groupNumber={31}
]{reportStyle}

\begin{document}

\maketitle

\section{Introduction}

The method used in this assignment is the use of the Density-Based Spatial Clustering of Applications with Noise (DBSCAN) algorithm. 
DBSCAN is a widely used data clustering algorithm which is commonly used in data mining and machine learning.
This algorithm is a density based clustering algorithm. 
It is a density based clustering algorithm because it uses the density of points in a region to determine if they belong to the same cluster. 
The algorithm is based on two parameters, the minimum number of points in a region and the maximum distance between points in a region. 

\section{Methods}

DBSCAN groups together points that are close to each other based on two parameters: $\epsilon$ (epsilon) and $minPts$.
$\epsilon$ is the maximum distance between two points to be considered close to each other.
The $minPts$ parameter is the minimum number of points that need to be close to a point for it to be considered a core point.
Every point that is not a core point but is inside the $\epsilon$ of a core point is considered a border point.
All connected core points and border points form a cluster. Every point that is neither a core point nor a border point is considered noise.
\\\\In six steps, the algorithm works as follows:

\begin{enumerate}
	\item For each point in the dataset, determine the number of points in a region around it.
	\item If the number of points in the region is larger or equal to the minimum number of points, the point is a core point.
	\item If the number of points in the region is smaller than the minimum number of points and it belongs to the neighbors of a core point, it becomes a border point.
	\item If the number of points in the region is smaller than the minimum number of points and it does not belong to the neighbors of a core point, it becomes a noise point.
	\item For each core point, determine the cluster it belongs to.
	\item For each border point, determine the cluster it belongs to.
\end{enumerate}
This algorithm has a few advantages. you can use it to find clusters of any shape and size.
There is no need to specify the number of clusters beforehand. And is also very robust to noise and outliers.
This algorithm works best when you choose the minimum number of points based on the size and noise of the dataset.
For two-dimensional data, the use of 4 as minimum number of points is recommended.
for higher dimensional data, the use of two times the dimensionality of the data is recommended.

\section{Experimental results}

\begin {figure}[H]
	\centering
	\includegraphics[width=0.8\textwidth]{../../code/output/dbscan_eps_0.0493_k_3.png}
	\caption{DBSCAN with $\epsilon=0.0493$ and $k=3$}
	\label{fig:dbscan_eps_0.0493_k_3}
\end {figure}

\autoref{fig:dbscan_eps_0.0493_k_3} shows the result of the DBSCAN algorithm with $\epsilon=0.0493$ and $k=3$.
The algorithm found multiple smaller clusters and multiple noise points.
The clusters are very close to each other and are not clearly visible in the figure.

\begin {figure}[H]
	\centering
	\includegraphics[width=0.8\textwidth]{../../code/output/elbow_plot_k_3.png}
	\caption{Elbow plot with $k=3$}
	\label{fig:elbow_plot_k_3}
\end {figure}

\autoref{fig:elbow_plot_k_3} shows the elbow plot with $k=3$.
The red dot is the elbow point determined by the researchers.
This should be the place with the greatest change in the slope of the curve.

\begin {figure}[H]
	\centering
	\includegraphics[width=0.8\textwidth]{../../code/output/dbscan_eps_0.071_k_4.png}
	\caption{DBSCAN with $\epsilon=0.071$ and $k=4$}
	\label{fig:dbscan_eps_0.071_k_4}
\end {figure}

\autoref{fig:dbscan_eps_0.071_k_4} shows the result of the DBSCAN algorithm with $\epsilon=0.071$ and $k=4$.
The algorithm found 3 clusters and multiple noise points.
The clusters are clearly visible in the figure.
The clusters are not perfectly circular, but they are still clearly visible.
The noise points are also clearly visible in the figure.

\begin {figure}[H]
	\centering
	\includegraphics[width=0.8\textwidth]{../../code/output/elbow_plot_k_4.png}
	\caption{Elbow plot with $k=4$}
	\label{fig:elbow_plot_k_4}
\end {figure}

\autoref{fig:elbow_plot_k_4} shows the elbow plot with $k=4$.
The red dot is the elbow point determined by the researchers.
This should be the place with the greatest change in the slope of the curve.

\begin {figure}[H]
	\centering
	\includegraphics[width=0.8\textwidth]{../../code/output/dbscan_eps_0.0748_k_5.png}
	\caption{DBSCAN with $\epsilon=0.0748$ and $k=5$}
	\label{fig:dbscan_eps_0.0748_k_5}
\end {figure}

\autoref{fig:dbscan_eps_0.0748_k_5} shows the result of the DBSCAN algorithm with $\epsilon=0.0748$ and $k=5$.
The algorithm found 3 clusters and multiple noise points.
The clusters are clearly visible in the figure.
This also shows outliers which are not clearly visible in the figure.

\begin {figure}[H]
	\centering
	\includegraphics[width=0.8\textwidth]{../../code/output/elbow_plot_k_5.png}
	\caption{Elbow plot with $k=5$}
	\label{fig:elbow_plot_k_5}
\end {figure}

\autoref{fig:elbow_plot_k_5} shows the elbow plot with $k=5$.
The red dot is the elbow point determined by the researchers.
This should be the place with the greatest change in the slope of the curve.

\begin {figure}[H]
	\centering
	\includegraphics[trim={0 5cm 0 5cm},clip,width=0.8\textwidth]{../../code/output/silhouette_scores.png}
	\caption{Silhouette scores}
	\label{fig:silhouette_scores}
\end {figure}

\autoref{fig:silhouette_scores} shows the silhouette scores for the different k values.

\section{Discussion}

The silhouette scores show that the best value for $k=5$. The diference between the scores for
$k=4$ and $k=5$ is very small.
The fact that the researchers determined the elbow points themselves is a disadvantage
and could have influenced the results.
So realistically there is not a definitive best value for k when comparing 4 and 5.
Furthermore, it advised to use 4 as the value for $k$, which could be a reason to choose $k=4$.
To conclude, the best value for $k$ is either $4$ or $5$, both values are equally as fair to use. 



\end{document}